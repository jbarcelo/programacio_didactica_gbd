\documentclass[catalan, a4paper, 12pt, titlepage]{article}
\usepackage{babel}
\usepackage{graphicx}
\usepackage{fontawesome5}
\graphicspath{ {./img/} }
\usepackage{mathptmx}
\renewcommand{\baselinestretch}{1.5}
%\usepackage{isolatin1}

%tikz
\usepackage{tikz}
\usetikzlibrary{mindmap}

%hyperref
%\usepackage[pdftex,pdfpagelabels,bookmarks,hyperindex,hyperfigures,hidelinks]{hyperref}

\usepackage{termcal}

% Few useful commands (our classes always meet either on Monday and Wednesday
% or on Tuesday and Thursday)

\newcommand{\MWClass}{%
\calday[Monday]{\classday} % Monday
\skipday % Tuesday (no class)
\calday[Wednesday]{\classday} % Wednesday
\skipday % Thursday (no class)
\skipday % Friday
\skipday\skipday % weekend (no class)
}

\newcommand{\TRClass}{%
\skipday % Monday (no class)
\calday[Tuesday]{\classday} % Tuesday
\skipday % Wednesday (no class)
\calday[Thursday]{\classday} % Thursday
\skipday % Friday
\skipday\skipday % weekend (no class)
}

\newcommand{\MTRFClass}{%
	\calday[Monday]{\classday}
	\calday[Tuesday]{\classday}
	\skipday % Dimecres
	\calday[Thursday]{\classday}
	\calday[Friday]{\classday}
\skipday\skipday % weekend (no class)
}

\newcommand{\Holiday}[2]{%
\options{#1}{\noclassday}
\caltext{#1}{#2}
}

\usepackage{lastpage}

\usepackage{fancyhdr}

\fancyhead[L]{Programació didàctica - Gestió de bases de dades}
\fancyhead[R]{Jaume Barceló Vicens}
\fancyfoot[C]{\small{Pàgina \thepage\ de \pageref{LastPage}}}

% Uncomment to remove the header rule
\renewcommand{\headrulewidth}{0pt}

\usepackage{enumitem}
\setitemize{noitemsep,topsep=0pt,parsep=0pt,partopsep=0pt}

\usepackage{pdfpages}

\title{Una programació didàctica de \\
gestió de bases de dades\\
	\includegraphics[width=10cm]{database.eps}
	}

\author{
	Jaume Barceló Vicens
	DNI 43135949R\\
	Cos 590 - Professors d'ensenyament secundari\\
	Especialitat 107 - Informàtica}

\date{
	Família professional informàtica \\
	Cicle Formatiu de Grau Superior d’Administració de Sistemes Informàtics en Xarxa\\
	Mòdul professional de gestió de bases de dades\\
	Codi 0372 \\
	170h (+90h d'anglès)\\
	%http://github.com/jbarcelo/programacio\_didactica\_gbd \\
	%\faCalendar*[regular] \today% \qquad  Llicència CC-BY-SA 4.0 \faCreativeCommons\ \faCreativeCommonsBy\ \faCreativeCommonsSa
	}

\setcounter{tocdepth}{1}

\begin{document}

\pagestyle{empty}

\maketitle

\tableofcontents 

\pagestyle{fancy}

\section{Introducció.}

Aquesta és la programació didàctica del mòdul ``Gestió de bases de dades'' (codi 0372) de primer curs del cicle formatiu ``Tècnic superior en administració de sistemes en xarxa'' que s'imparteix al CIFP Francesc de Borja Moll el curs escolar 2021-2022.
%La programació didàctica es presenta com una guia oberta i flexible, sobre la que es poden fer tants canvis com siguin necessaris per adaptar-se a la realitat del grup d'alumnes i del curs en general.

El document s'organitza en seccions. 
La secció \ref{sec:normativa} recull el marc normatiu.
La secció \ref{sec:contextualització} descriu el context en el qual s'impartirà el mòdul tractant aspectes com l'ensenyança dual, l'ensenyança en llengua estrangera i el centre integrat de formació professional.
A continuació, la secció \ref{sec:continguts} detalla els objectius, competències i continguts tal com apareixen a la normativa.
La concreció d'aquests continguts arribarà més endavant a l'hora de descriure les unitats didàctiques.

Els procediments d'avaluació s'introdueixen de manera general a la secció \ref{sec:procediments} i els instruments d'avaluació a la secció \ref{sec:instruments}.
L'objectiu de l'avaluació és la millora, i els instruments d'avaluació ens permeten mesurar el grau d'assoliment dels objectius i detectar problemes.

Les activitats es descriuen de manera genèrica a la secció \ref{sec:activitats}.
En general inclouen: explicacions del professor, autoaprenentatge i aprenentatge grupal, tasques avaluables i exàmens.

La metodologia descrita a la secció \ref{sec:metodologia} s'ajusta a la realitat del centre i del grup.
En particular, es tracta d'una metodologia centrada en l'alumne.
L'alumne serà protagonista i responsable del procés d'ensenyament-aprenentatge.
Tot seguit, a la secció \ref{sec:activitats_coherents} descriurem com són aquestes activitats coherents amb la metodologia proposada.
També es descriu el model d'activitat oberta, que adapta el nivell de dificultat a l'estudiant.

La distribució temporal de les unitats didàctiques es troba a la secció \ref{sec:distribució} i el detall de les unitats didàctiques, que particularitza tots els aspectes descrits anteriorment, a la secció \ref{sec:unitats}.

A la secció \ref{sec:diversitat} explicarem la nostra manera de treballar l'atenció a la diversitat, que és un dels aspectes més importants del procés d'ensenyament-aprenentatge.

La secció \ref{sec:recursos_i_materials} explica el material i recursos que usarem en aquest mòdul.
Hi ha material de creació pròpia, un llibre de referència, diapositives, activitats, etc.

Finalment, les conclusions es troben a la secció \ref{sec:conclusió}.

El calendari orientatiu s'inclou a l'apèndix \ref{app:schedule}.

\section{Normativa.}
\label{sec:normativa}

Aquest mòdul està regulat per la següent normativa estatal:
\begin{itemize}
	\item Llei Orgànica 2/2006, de 3 de maig, d'Educació, que assenyala que el Govern d'Espanya, prèvia consulta a les comunitats autònomes, establirà les titulacions de formació professional i els aspectes bàsics del currículum.
	\item Llei Orgànica 5/2002, de 19 de juny, de les Qualificacions i de la Formació Professional, que posa en marxa el Sistema Nacional de Qualificacions i Formació Professional, desenvolupada pel Reial decret 1128/2003, de 5 de setembre, modificat pel Reial decret 1416/2005, de 25 de novembre, sobre el Catàleg Nacional de Qualificacions Professionals.
	\item Reial decret 1147/2011, de 29 de juliol, pel qual s'estableix l'ordenació general de la formació professional basada en el Catàleg Nacional de Qualificacions Professionals.
	\item Reial decret 1629/2009, de 30 d'octubre, pel qual s'estableix el títol de ``Tècnic Superior en Administració de Sistemes Informàtics en Xarxa'' i es fixen els seus ensenyaments mínims.
	\item Ordre EDU/392/2010, de 20 de gener, per la qual s'estableix el currículum del cicle formatiu de grau superior corresponent al títol de ``Tècnic Superior en Administració de Sistemes Informàtics en Xarxa''.
\end{itemize}

En l'àmbit autonòmic:
\begin{itemize}
	\item Decret 91/2012, de 23 de novembre, pel qual s'estableix l'ordenació general de la formació professional del sistema educatiu en el sistema integrat de formació professional a les Illes Balears.
	\item Ordre de la consellera d'Educació i Cultura de 13 de juliol de 2009 per la qual es regula l'organització i el funcionament dels cicles formatius de formació professional del sistema educatiu que s'imparteixen d'acord amb la Llei orgànica 2/2006, de 3 de maig, d'educació, a les Illes Balears, en la modalitat d'ensenyament presencial.
	\item Resolució del conseller d’Educació i Formació Professional de 15 d’abril de 2021 per la qual s’estableix el calendari escolar del curs 2021-2022 per als centres docents no universitaris de la comunitat autònoma de les Illes Balears.
\end{itemize}

%Finalment, en l'àmbit del centre:
%\begin{itemize}
%	\item Projecte educatiu de centre (PEC) que planteja les intencions educatives i organitzatives del CIFP: la seva missió, visió i valors. I que inclou com a annexos el projecte lingüístic, el reglament d'organització i funcionament i la concreció curricular del centre.
%\end{itemize}

\section{Contextualització.}
\label{sec:contextualització}

Aquesta programació didàctica és per al mòdul Gestió de Bases de Dades del Cicle Formatiu de Grau Superior d'Administració de Sistemes Informàtics en Xarxa del Centre Integrat de Formació Professional Francesc de Borja Moll. 

A la comunitat Autònoma de les Illes Balears, la tipologia dels centres integrats ha estat reconeguda a partir de la Resolució del conseller d'educació i universitat de 24 de maig de 2018 per la qual es determina la tipologia dels centres docents públics no universitaris.

El centre es crea el curs 2019-20 amb estudis de doble torn i comparteix espais amb l'IES Nou Llevant. A més d'informàtica, hi trobam estudis de comerç i màrqueting, imatge personal, sanitat, i seguretat i medi ambient.

El CIFP Francesc de Borja Moll ofereix formació professional bàsica, el grau mitjà i els tres graus superiors de la família professional d'informàtica. 
A més, ofereix programes d'FP dual tant per a ASIX com per a DAW.
La gran oferta formativa d'informàtica fa que aquest departament sigui molt nombrós, amb tots els avantatges i inconvenients que comporta.
Per una banda, hi ha una gran quantitat de coneixement i expertesa acumulats que permet aprendre i fer projectes que serien més difícils en un departament més petit.
Per l'altra, a vegades pot ser difícil coordinar i posar d'acord tanta gent.

Un dels avantatges que suposa estar en un centre gran és que és possible per al professor impartir el mateix mòdul a diferents grups i, per tant, concentrar esforços.
Per exemple, el mateix professor pot tenir els grups d'ASIX i ASIX dual.

\subsection{Aprenentatge-ensenyament en anglès.}

Es tracta d'un mòdul en anglès i això pot representar una dificultat addicional per a qualcuns estudiants.
El professor haurà de ser comprensiu, traduir tot allò que sigui necessari i animar als alumnes que s'ajudin entre ells per afrontar aquest repte.
L'anglès és molt necessari per a un professional de la informàtica, ja que es tracta d'un àmbit molt globalitzat.
Encara més en un context turístic com el de les Illes Balears.
Per aquest motiu, tot el temps dedicat a l'ensenyament-aprenentatge de l'anglès és un temps ben invertit.
Aquesta programació ja presenta qualcunes seccions en anglès, en concordança amb allò que serà el mòdul.

\subsection{Aprenentatge-ensenyament dual.}

El grup d'ASIX dual presenta certes particularitats, ja que s'ha d'incorporar a la feina a mig curs i, en conseqüència, reduir el temps que passa a l'institut.
L'alumnat haurà d'aprendre al lloc de feina part dels continguts i les competències del mòdul.
Això vol dir que el professor haurà de ser flexible en el sentit que allò que fan els estudiants al centre de treball escapa, en part, al seu control.

A més, durant els primers mesos de classe s'ha de proporcionar a l'alumnat de dual la formació necessària perquè pugui afrontar amb èxit les entrevistes i la incorporació al lloc de feina.
Al mateix temps, s'ha de tenir prou flexibilitat per redirigir el curs en cas que la incorporació a l'empresa no sigui possible per qualque esdeveniment inesperat.

\section{Objectius. Competències. Continguts.}
\label{sec:continguts}

\subsection{Objectius generals.}

Els objectius generals estableixen les capacitats que s'espera que els alumnes hagin assolit a final de curs. Segons el Títol de Tècnic Superior en Administració de Sistemes Informàtics en Xarxa, s'estableixen els objectius de la següent llista.
El reial decret especifica que el mòdul de bases de dades contribueix a assolir els objectius generals d), e) i m) [sic].

1. Analitzar l'estructura del programari de base, comparant les característiques i prestacions de sistemes lliures i propietaris, per administrar sistemes operatius de servidor.

2. Instal·lar i configurar el programari de base, seguint documentació tècnica i especificacions donades, per administrar sistemes operatius de servidor.

3. Instal·lar i configurar programari de missatgeria i transferència de fitxers, entre d'altres, relacionant-los amb la seva aplicació i seguint documentació i especificacions donades, per administrar serveis de xarxa.

\faCheck 4. Instal·lar i configurar programari de gestió, seguint especificacions i analitzant entorns d'aplicació, per administrar aplicacions.

\faCheck 5. Instal·lar i administrar programari de gestió, relacionant-lo amb la seva explotació, per implantar i gestionar bases de dades.

6. Configura dispositius de maquinari, analitzant les seves característiques funcionals, per optimitzar el rendiment del sistema.

7. Configura maquinari de xarxa, analitzant les seves característiques funcionals i relacionant-lo amb el seu camp d'aplicació, per a integrar equips de comunicacions.

8. Analitzar tecnologies d'interconnexió, descrivint les seves característiques i possibilitats d'aplicació, per configurar l'estructura de la xarxa telemàtica i avaluar el seu rendiment.

9. Elaborar esquemes de xarxes telemàtiques utilitzant programari específic per a configurar l'estructura de la xarxa telemàtica.

10. Seleccionar sistemes de protecció i recuperació, analitzant les seves característiques funcionals, per posar en marxa solucions d'alta disponibilitat.

11. Identificar condicions d'equips i instal·lacions, interpretant plans de seguretat i especificacions del fabricant, per supervisar la seguretat física.

12. Aplicar tècniques de protecció contra amenaces externes, tipificant-les i avaluant-les per assegurar el sistema.

\faCheck 13. Aplicar tècniques de protecció contra pèrdues d'informació, analitzant plans de seguretat i necessitats d'ús per assegurar les dades.

14. Assignar els accessos i recursos de sistema, aplicant les especificacions de l'explotació, per administrar usuaris.

15. Aplicar tècniques de monitoratge interpretant els resultats i relacionant-los amb les mesures correctores per diagnosticar i corregir les disfuncions.

16. Establir la planificació de tasques, analitzant activitats i càrregues de treball de sistema per gestionar el manteniment.

17. Identificar els canvis tecnològics, organitzatius, econòmics i laborals en la seva activitat, analitzant les seves implicacions en l'àmbit de treball, per resoldre problemes i mantenir una cultura d'actualització i innovació.

18. Identificar formes d'intervenció en situacions col·lectives, analitzant el procés de presa de decisions i efectuant consultes per liderar les mateixes.

19. Identificar i valorar les oportunitats d'aprenentatge i la seva relació amb el món laboral, analitzant les ofertes i demandes de mercat per gestionar la seva carrera professional.

20. Reconèixer les oportunitats de negoci, identificant i analitzant demandes de mercat per crear i gestionar una petita empresa.

21. Reconèixer els seus drets i deures com a agent actiu en la societat, analitzant el marc legal que regula les condicions socials i laborals per participar com a ciutadà democràtic.


\subsection{Competències professionals, personals i socials.}

Les competències professionals, personals i socials d'aquest títol són les que es relacionen a continuació. 
El reial decret especifica que el mòdul de bases de dades contribueix a assolir les competències c), d) i m) [sic].

1. Administrar sistemes operatius de servidor, instal·lant i configurant el programari, en condicions de qualitat per assegurar el funcionament del sistema.

2. Administrar serveis de xarxa (web, missatgeria electrònica i transferència d'arxius, entre d'altres) instal·lant i configurant el programari, en condicions de qualitat.

\faCheck 3. Administrar aplicacions instal·lant i configurant el programari, en condicions de qualitat per respondre a les necessitats de l'organització.

\faCheck 4. Implantar i gestionar bases de dades instal·lant i administrant el programari de gestió en condicions de qualitat, segons les característiques de l'explotació.

5. Optimitzar el rendiment de sistema configurant els dispositius de maquinari d'acord amb els requisits de funcionament.

6. Avaluar el rendiment dels dispositius de maquinari identificant possibilitats de millores segons les necessitats de funcionament.

7. Determinar la infraestructura de xarxes telemàtiques elaborant esquemes i seleccionant equips i elements.

8. Integrar equips de comunicacions en infraestructures de xarxes telemàtiques, determinant la configuració per assegurar la seva connectivitat.

9. Implementar solucions d'alta disponibilitat, analitzant les diferents opcions de mercat, per protegir i recuperar el sistema davant de situacions imprevistes.

10. Supervisar la seguretat física segons especificacions de fabricant i el pla de seguretat per evitar interrupcions en la prestació de serveis de sistema.

11. Assegurar el sistema i les dades segons les necessitats d'ús i les condicions de seguretat establertes per prevenir fallades i atacs externs.

12. Administrar usuaris d'acord amb les especificacions d'explotació per garantir els accessos i la disponibilitat dels recursos de sistema.

\faCheck 13. Diagnosticar les disfuncions de sistema i adoptar les mesures correctives per restablir la seva funcionalitat.

14. Gestionar i / o realitzar el manteniment dels recursos de la seva àrea (programant i verificant el seu compliment), en funció de les càrregues de treball i el pla de manteniment.

15. Efectuar consultes, dirigint-se a la persona adequada i saber respectar l'autonomia dels subordinats, informant quan sigui convenient.

16. Mantenir l'esperit d'innovació i actualització en l'àmbit del seu treball per adaptar-se als canvis tecnològics i organitzatius del seu entorn professional.

17. Liderar situacions col·lectives que es puguin produir, intervenint en conflictes personals i laborals, contribuint a l'establiment d'un ambient de treball agradable i actuant en tot moment de forma sincera, respectuosa i tolerant.

18. Resoldre problemes i prendre decisions individuals, seguint les normes i procediments establerts, definits dins de l'àmbit de la seva competència.

19. Gestionar la seva carrera professional, analitzant les oportunitats d'ocupació, autoocupació i d'aprenentatge.

20. Participar de manera activa en la vida econòmica, social i cultural amb actitud crítica i responsable.

21. Crear i gestionar una petita empresa, realitzant un estudi de viabilitat de productes, de planificació de la producció i de comercialització.


\subsection{Continguts i Orientacions Pedagògiques.}

\subsubsection{Continguts bàsics.}

Sistemes d'emmagatzematge de la informació:

- Fitxers (plans, indexats i accés directe, entre altres).

- Bases de dades. Conceptes, usos i tipus segons el model de dades, la ubicació de la informació.

- Sistemes gestors de base de dades: funcions, components i tipus.

Disseny lògic de bases de dades:

- Model de dades.

- La representació del problema: els diagrames E / R entitats i relacions. Cardinalitat. Debilitat.

- El model E / R ampliat.

- El model relacional: Terminologia del model relacional. Característiques d'una relació. Claus primàries i claus alienes.

- Pas del diagrama E / R al model relacional.

- Normalització.

Disseny físic de bases de dades:

- Eines gràfiques proporcionades pel sistema gestor per a la implementació de la base de dades.

- El llenguatge de definició de dades.

- Creació, modificació i eliminació de bases de dades.

- Creació, modificació i eliminació de taules. Tipus de dades.

- Implementació de restriccions.

Realització de consultes:

- Eines gràfiques proporcionades pel sistema gestor per a la realització de consultes.

- La sentència SELECT.

- Selecció i ordenació de registres. Tractament de valors nuls.

- Consultes de resum. Agrupament de registres.

- Unió de consultes.

- Composicions internes i externes.

- Subconsultes.

Edició de les dades:

- Eines gràfiques proporcionades pel sistema gestor per a l'edició de la informació.

- Les sentències INSERT, DELETE i UPDATE.

- Subconsultes i combinacions en ordres d'edició.

- Transaccions. Sentències de processament de transaccions.

- Accés simultani a les dades: polítiques de bloqueig.

Construcció de guions:

- Introducció. Llenguatge de programació.

- Tipus de dades, identificadors, variables.

- Operadors. Estructures de control.

Gestió de la seguretat de les dades:

- Recuperació de fallades.

- Còpies de seguretat.

- Eines gràfiques i utilitats proporcionades pel sistema gestor per a la realització i recuperació de còpies de seguretat.

- Sentències per a la realització i recuperació de còpies de seguretat.

- Eines gràfiques i utilitats per a importació i exportació de dades.

- Transferència de dades entre sistemes gestors.

\subsubsection{Orientacions pedagògiques.}

Aquest mòdul professional conté la formació necessària per exercir la funció de gestor de bases de dades.

La gestió de bases de dades inclou aspectes com:

- La planificació i realització del disseny físic d'una base de dades.

- La inserció i manipulació de dades.

- La planificació i realització de consultes.

- La planificació i execució d'importacions, exportacions i migracions de dades.

- La planificació i aplicació de mesures d'assegurament de la informació.

Les activitats professionals associades a aquesta funció s'apliquen en:

- La implantació de bases de dades.

- La gestió de la informació emmagatzemada en bases de dades.

Les línies d'actuació en el procés d'ensenyament-aprenentatge que permeten assolir els objectius versaran sobre:

- La interpretació de dissenys lògics de bases de dades.

- La realització del disseny físic d'una base de dades a partir d'un disseny lògic.

- La implementació de bases de dades.

- La realització d'operacions amb dades emmagatzemades.

- La importació i exportació de dades.

- L'assegurament de la informació.

\subsection{Connexió amb altres mòduls.}

Tot i que el cicle es divideix en mòduls de temàtica diferent, les competències i els continguts estan interconnectats.

\begin{itemize}
	\item Planificació i administració de xarxes. Els alumnes necessiten conèixer els conceptes d'internet, xarxa, client, servidor, adreça IP i port. Sovint els sistemes de bases de dades funcionen en xarxa seguint una arquitectura client-servidor.
	\item Implantació de Sistemes Operatius. Els alumnes necessiten conèixer les comandes bàsiques del sistema operatiu per a instal·lar i configurar sistemes gestors de bases de dades i llavors connectar-se amb un client. També necessiten coneixements de virtualització i de contenidors.
	\item Fonaments de hardware. Els alumnes necessiten saber els requisits de maquinari del programari que s'utilitza.
	\item Llenguatge de Marques i Sistemes de Gestió de la Informació. Existeixen bases de dades que utilitzen el format XML o JSON. Els alumnes necessiten conèixer la possibilitat d'emmagatzemar i intercanviar dades en format XML o JSON.
	\item Formació i Orientació Laboral. Els alumnes han de saber a quins perfils professionals es correspon la formació del mòdul, en especial el de Database Administrator o DBA. Cal tenir una noció orientativa del nivell de responsabilitat i també de condicions de feina i remuneració.
	\item Implantació de Sistemes Operatius i Administració de Sistemes Operatius. Els sistemes gestors de bases de dades s'executen sobre un sistema operatiu que cal conèixer.
	\item Serveis en Xarxa i Internet. És habitual que els serveis de bases de dades s'ofereixin a través de la xarxa. A més, poden formar part de sistemes connectats a internet i, per tant, s'han de conèixer les tecnologies dels serveis en xarxa i internet.
	\item Administració de Sistemes Gestors de Bases Dades. Els SGBD (o DBMS en anglès) són els motors que permeten l'existència de les bases de dades. Una vegada es coneixen els conceptes bàsics de bases de dades, és necessari aprendre el funcionament dels SGBD.
	\item Seguretat i Alta Disponibilitat. Les dades són un dels actius més importants de les organitzacions. S'han de prendre tota classe de mesures per a protegir aquestes dades i perquè estiguin en disponibles en tot moment.
	\item Empresa i iniciativa emprenedora. Un coneixement profund de les bases de dades permet als alumnes pensar en la possibilitat de crear empreses per a oferir serveis a tercers. Aquests serveis poden ser també en la forma de consultoria.
\end{itemize}

\subsection{Elements transversals.}

S'han elegit uns elements transversals per a treballar en aquest mòdul.

\subsubsection{Respecte pels valors cívics.}

Es començarà per l'exemple del professor i es fomentarà el tracte igualitari i el respecte mutu. També són importants les bones pràctiques professionals i la transparència perquè la feina dels informàtics sigui beneficiosa per al conjunt de la comunitat.

\subsubsection{Hàbits de vida saludable.}

S'evitaran en la mesura possible les situacions estressants i es facilitarà un bon ambient de feina on els alumnes puguin tenir els descansos necessaris per assegurar un bon rendiment i una bona salut.

\subsubsection{Foment de la lectura.}

Els estudiants hauran de ser els protagonistes del procés d'ensenyament aprenentatge i això significa que hauran de llegir molt de material més enllà del que ofereix el professor. Es tractarà, en la seva majoria, de material tècnic.

\subsubsection{Millora de la capacitat d'expressió oral i escrita.}

Es valoraran els esforços de l'alumnat per millorar l'expressió oral i escrita. Especialment si ho fan en anglès, que és l'idioma de l'assignatura.

\section{Procediments d'avaluació.}
\label{sec:procediments}

L'avaluació \cite{coll2017} són les accions del procés d'ensenyament-aprenentatge que:

\begin{itemize}
	\item determinen el grau d'assoliment dels objectius i les competències.
	\item detecten les dificultats i els errors del procés.
	\item suggereixen camins de millora.
\end{itemize}

Els procediments d'avaluació descrits a la Subsecció \ref{subsec:procediments} s'usaran en combinació amb els criteris d'avaluació descrits a la normativa.

\subsection{Resultats d'aprenentatge i criteris d'avaluació.}
\label{subsec:resultats}.

Els resultats d'aprenentatge i els seus corresponents criteris d'avaluació del mòdul de Gestió de Bases de Dades són els següents.

1. Reconeix els elements de les bases de dades analitzant les seves funcions i valorant la utilitat de sistemes gestors.

Criteris d'avaluació:

a) S'han analitzat els diferents sistemes lògics d'emmagatzematge i les seves funcions.

b) S'han identificat els diferents tipus de bases de dades segons el model de dades utilitzat.

c) S'han identificat els diferents tipus de bases de dades en funció de la ubicació de la informació.

d) S'ha reconegut la utilitat d'un sistema gestor de bases de dades.

e) S'ha descrit la funció de cada un dels elements d'un sistema gestor de bases de dades.

f) S'han classificat els sistemes gestors de bases de dades.

2. Dissenya models lògics normalitzats interpretant diagrames entitat / relació.

Criteris d'avaluació:

a) S'ha identificat el significat de la simbologia pròpia dels diagrames entitat / relació.

b) S'han utilitzat eines gràfiques per a representar el disseny lògic.

c) S'han identificat les taules del disseny lògic.

d) S'han identificat els camps que formen part de les taules del disseny lògic.

e) S'han identificat les relacions entre les taules del disseny lògic.

f) S'han definit els camps clau.

g) S'han aplicat les regles d'integritat.

h) S'han aplicat les regles de normalització fins a un nivell adequat.

i) S'han identificat i documentat les restriccions que no poden plasmar-se en el disseny lògic.

3. Realitza el disseny físic de bases de dades utilitzant assistents, eines gràfiques i el llenguatge de definició de dades.

Criteris d'avaluació:

a) S'han definit les estructures físiques d'emmagatzematge.

b) S'han creat taules.

c) S'han seleccionat els tipus de dades adequats.

d) S'han definit els camps clau en les taules.

e) S'han implantat totes les restriccions reflectides en el disseny lògic.

f) S'ha verificat mitjançant un conjunt de dades de prova que la implementació s'ajusta al model.

g) S'han utilitzat assistents i eines gràfiques.

h) S'ha utilitzat el llenguatge de definició de dades.

i) S'ha definit i documentat el diccionari de dades.

4. Consulta la informació emmagatzemada manejant assistents, eines gràfiques i el llenguatge de manipulació de dades.

Criteris d'avaluació:

a) S'han identificat les eines i sentències per realitzar consultes.

b) S'han realitzat consultes simples sobre una taula.

c) S'han realitzat consultes que generen valors de resum.

d) S'han realitzat consultes sobre el contingut de diverses taules mitjançant composicions internes.

e) S'han realitzat consultes sobre el contingut de diverses taules mitjançant composicions externes.

f) S'han realitzat consultes amb subconsultes.

g) S'han valorat els avantatges i inconvenients de les diferents opcions vàlides per dur a terme una consulta determinada.

5. Modifica la informació emmagatzemada utilitzant assistents, eines gràfiques i el llenguatge de manipulació de dades.

Criteris d'avaluació:

a) S'han identificat les eines i sentències per a modificar el contingut de la base de dades.

b) S'han inserit, esborrat i actualitzat dades en les taules.

c) S'ha inclòs en una taula la informació resultant de l'execució d'una consulta.

d) S'han adoptat mesures per a mantenir la integritat i consistència de la informació.

e) S'han dissenyat guions de sentències per dur a terme tasques complexes.

f) S'ha reconegut el funcionament de les transaccions.

g) S'han anul·lat parcialment o totalment els canvis produïts per una transacció.

h) S'han identificat els efectes de les diferents polítiques de bloqueig de registres.

6. Executa tasques d'assegurament de la informació, analitzant-les i aplicant mecanismes de salvaguarda i transferència.

Criteris d'avaluació:

a) S'han identificat eines gràfiques i en línia de comandes per a l'administració de còpies de seguretat.

b) S'han realitzat còpies de seguretat.

c) S'han restaurat còpies de seguretat.

d) S'han identificat les eines per a importar i exportar dades.

e) S'han exportat dades a diversos formats.

f) S'han importat dades amb diferents formats.

g) S'ha interpretat correctament la informació subministrada pels missatges d'error i els fitxers de registre.

h) S'ha transferit informació entre sistemes gestors.

\subsection{Procediments d'avaluació per als criteris d'avaluació.}
\label{subsec:procediments}

Al procediment clàssic on el professor avalua a l'alumne afegirem:

\begin{itemize}
	\item Autoavaluació: Es proposaran exercicis on els alumnes podran comprovar si l'han resolt bé o no.
	\item Avaluació entre iguals: Els alumnes examinen les respostes dels companys i aporten les seves opinions.
	\item Avaluació 360: Els alumnes avaluen el procés d'ensenyament i fan propostes constructives.
\end{itemize}

\section{Instruments d'avaluació.}
\label{sec:instruments}

L'avaluació permet fer seguiment del progrés dels alumnes i és contínua.
Això vol dir que hi haurà moltes activitats avaluables i que els alumnes aniran rebent informació relativa a la seva evolució i les expectatives del professor.
La correcció de l'activitat per part del professor s'ha de fer en pocs dies a partir de l'entrega de l'activitat, perquè l'avaluació tengui la seva màxima efectivitat.

Les activitats avaluables es divideixen en dos grans grups. 
Les tasques de classe i les proves o exàmens individuals.

Les tasques de classe sovint es fan per parelles, amb l'ajuda de tota la classe.
Els estudiants tenen temps per cercar informació, consultar als seus companys i professors i fer proves i experimentar. 
Els estudiants disposen de temps a classe per a completar les tasques, però pot ser que aquells estudiants que necessiten més temps també hagin de treballar a casa.
Aquestes tasques tenen com a objectiu fonamental l'anomenat ``aprendre fent'' (``learn by doing'').
Els alumnes tenen l'oportunitat de posar en pràctica allò que ha explicat el professor o fins i tot aprendre ells mateixos com a grup, amb el professor únicament com a suport per a resoldre dubtes i guiar l'aprenentatge.
Moltes d'aquestes tasques consisteixen en l'elaboració d'un informe o un tutorial.
Això permet als estudiants desenvolupar habilitats de comunicació, identificar els aspectes clau del tema tractat i pensar de manera profunda sobre allò que estan treballant. Els informes generats podran ser d'utilitat més endavant quan l'estudiant necessita refrescar allò que va fer, o fins i tot podran ser consultats en alguns exàmens.
Les tasques poden ser llargues i dur-se a terme al llarg de múltiples sessions.

Les proves o exàmens individuals es fan al final de cada unitat didàctica en silenci i sense parlar amb els companys i, en general, sense ajuda del professor.
A més, es fan en una sola sessió i amb un temps limitat.
Aquestes proves consisteixen en exercicis similars a aquells que s'han desenvolupat durant la unitat didàctica i permeten valorar si cada un dels alumnes ha après allò que ha treballat de manera grupal.
Aquestes proves serveixen de control per assegurar que cada un dels integrants del grup ha participat en les tasques i ha assolit els objectius marcats.
Hi ha diferents tipologies d'exàmens individuals, per cobrir un espectre més gran d'avaluació. Aquests exàmens poden ser de tipus test, de captures de pantalla o en paper. 
Alguns són amb accés lliure a internet i al material d'estudi però amb temps molt limitat.
En uns altres, els alumnes disposen de molt de temps per pensar, però no poden accedir a internet ni consultar material de classe.

Les proves tipus test presenten a cada pregunta quatre opcions diferents de les quals l'alumne n'ha d'elegir únicament una.
S'aplica una correcció a la nota per tenir en compte que responent aleatòriament s'encertarien un 25\% de les respostes.

En les proves de captura de pantalla, els alumnes han d'acomplir un procés o seguir unes passes.
Per demostrar que el saben dur a terme, han d'anar fent captures de pantalla que entregaran al professor abans que acabi l'examen.

Finalment, els exàmens en paper es fan sense fer servir l'ordinador i els alumnes hauran de resoldre exercicis i respondre preguntes fent servir un bolígraf.

La taula \ref{tab:tasquesiexamens} resumeix les diferències entre les tasques de classe i els exàmens.

\begin{table}
	\small
        \centering
        \begin{tabular}{lr}
        Tasques & Exàmens\\
        \hline
	Per parelles & Individuals\\
		Es pot parlar & En silenci \\
		Amb ajuda de tota la classe &Sense ajuda dels companys\\
		Amb ajuda del professor & Ajuda del professor limitada\\
		Poden allargar-se més d'una sessió & En una única sessió\\
		Poden acabar-se a casa & S'han de fer a classe\\
		Durant una UD & Al final d'una UD \\
		Exploren nous continguts & Sobre temes que ja s'jan treballat\\
		Per a aprendre & Per consolidar i demostrar allò que s'ha après\\
		Elaboració d'informes i tutorials & Test, captures de pantalla o en paper.

\end{tabular}
        \caption{Diferències entre tasques i exàmens} \label{tab:tasquesiexamens}
\end{table}

Per a cada activitat avaluable s'oferirà a l'alumne una nota entre el 0 i el 100.
Aquestes notes s'utilitzaran per a calcular la nota final de l'alumne.
El conjunt de tasques i el conjunt d'exàmens tenen el mateix pes.
En qualsevol cas, l'avaluació serà un procés individualitzat on el professor tendrà en compte tota la interacció amb l'alumnat durant el curs així com els criteris d'avaluació descrits a la Subsecció \ref{subsec:resultats}.

%\begin{figure}
%\centering
%\begin{tikzpicture}[mindmap, grow cyclic, every node/.style=concept, concept color=orange!40, 
%	level 1/.append style={level distance=5cm,sibling angle=45},
%	level 2/.append style={level distance=3cm,sibling angle=45},]
%
%\node{Exàmens}
%	  child{ node {Suport}
%	    child{ node {Ordinador}}
%	    child{ node {Paper}}
%	  }
%	  child{ node {Respostes}
%	    child{ node {Tancada}}
%	    child{ node {Oberta}}
%	    child{ node {Screenshot}}
%	  }
%;
%\end{tikzpicture}
%\caption{Classificació dels exàmens individuals segons el tipus de suport i el tipus de resposta.} \label{fig:M1}
%\end{figure}

Per a aquells alumnes que no aprovin l'avaluació contínua es preparà un examen de recuperació ordinària i un examen de recuperació extraordinària. Cada un d'aquests exàmens tracta tot allò que s'ha fet durant el mòdul i, per tant, són exigents a causa del seu abast. Els alumnes han de saber el mecanisme habitual per aprovar el mòdul és l'avaluació contínua.

\subsection{Avaluació de la tasca docent.}

De manera trimestral es demanarà als alumnes que reflexionin sobre el mòdul i la tasca docent. Aquesta informació s'utilitzarà per a intentar reorientar i millorar el mòdul en els trimestres següents i en les pròximes edicions del curs. 

\begin{itemize}
	\item Què és el més interessant que hem vist?
	\item Què és allò que ha quedat menys clar o que genera més dubtes?
	\item Què en penses del mòdul i de la tasca del professor?
	\item Què es podria fer per millorar?
\end{itemize}



\section{Activitats per al desenvolupament de competències.}
\label{sec:activitats}

Les activitats per al desenvolupament de competències inclouen els processos d'ensenyament-aprenentatge, autoavaluació i avaluació per part del professor.

Aquests són exemples d'activitats:
\begin{itemize}
	\item Explicacions del professor. Amb el suport de diapositives, projector o pissarra. Oferiran una introducció i context perquè llavors els alumnes puguin emprendre altres activitats.
	\item Activitats d'autoaprenentatge i autoavaluació. Els alumnes, per parelles i com a grup, utilitzaran material que els permetrà adquirir els resultats d'aprenentatge. Aquestes activitats no són avaluables. Són una preparació per a altres activitats.
	\item Tasques avaluables. Els alumnes, per parelles, realitzaran tasques que entregaran al professor per a l'avaluació i qualificació. A vegades també tendrem avaluació entre iguals.
	\item Repàs. És una activitat que típicament precedeix un examen.
És precisament abans de l'examen quan els alumnes tenen la màxima motivació per resoldre els dubtes.
Aprofitam aquesta situació per assegurar que els continguts de la unitat didàctica s'han assimilat i els resultats d'aprenentatge s'han assolit.
	\item Examen. És una activitat individual per controlar que cada alumne ha assolit els resultats d'aprenentatge.
	\item Correcció en grup. És una activitat que típicament segueix un examen o una activitat d'autoavaluació. Entre tota la classe es trobaran les respostes correctes amb la supervisió del professor. En acabar un examen, els alumnes es senten motivats per esbrinar quines són les respostes correctes i comparar-les amb les seves. Aprofitarem aquesta motivació per fer un darrer repàs.

\end{itemize}

La majoria de les tasques de classe són per parelles o grups, mentre que els exàmens són individuals.
A la secció \ref{sec:instruments} ja hem detallat com són els exàmens.

A més, es reserven els primers i darrers 5 minuts de cada sessió de 110 minuts per a aspectes més informals.
Per exemple, consulta de dubtes amb el professor, recollida de retroalimentació (``feedback'') sobre el funcionament del curs, anotació en un diari de classe sobre les activitats realitzades aquell dia o programades per a la sessió següent, etc.

La secció \ref{sec:unitats} detalla les activitats de cada una de les unitats didàctiques mentre que el calendari orientatiu que ens indica quan es farà cada activitat es troba a l'apèndix \ref{app:schedule}.

\section{Metodologia.}
\label{sec:metodologia}

Segons el Real Decret 1147/2011, de 29 de juliol, pel qual s'estableix l'ordenació general de formació professional del sistema educatiu, la metodologia didàctica de les ensenyances de formació professional integrarà els aspectes científics, tecnològics i organitzatius que corresponguin en cada cas, amb la finalitat que l'alumnat adquireixi una visió global dels processos productius de l'activitat professional corresponent.

Es prioritzaran aquelles metodologies en les quals l'alumne sigui el protagonista. Especialment, l'acompliment de tasques per parelles en les quals els alumnes puguin ``aprendre fent'' (learn by doing) i aprendre també dels seus companys.

El professor farà breus presentacions seguint el format de classe magistral quan sigui imprescindible, habitualment per a la introducció i la contextualització d'un tema.
Sempre que sigui possible, els alumnes aprendran fent tasques, discutint-les i presentant-les.
Així mateix, s'espera que durant aquestes tasques sorgeixin dubtes. 
A vegades serà possible per als alumnes resoldre'ls de manera autònoma i consultant els companys, però quan això no sigui possible el professor ha d'intervenir fent totes les explicacions i aclariments que siguin necessaris per acompanyar als alumnes pel bon camí de l'aprenentatge.

Es fomentarà el treball en parelles, ja que el treball en equip és una habilitat fonamental que s'ha de practicar.
A més, permetent que els alumnes treballin per parelles es poden presentar tasques d'un nivell de dificultat més alt i els alumnes avancen més ràpidament, ja que si un s'equivoca o no sap com continuar, probablement l'altre sí que ho farà bé.

Que el treball sigui per parelles no significa que no pugui treballar tota la classe com un únic equip.
Al contrari, s'animarà als alumnes que s'ajudin uns grups als altres i fins i tot que prenguin temporalment el rol de professor per explicar qualque cosa a tota la classe.
De totes maneres cada parella es responsabilitza de fer la seva tasca de manera completa i d'entregar un informe detallat documentant el procés.

No és acceptable que una parella es divideixi la pràctica en dues parts i cada membre treballi de manera independent sense conèixer allò que fa l'altre. 
Cada integrant de la parella ha de tenir un coneixement total de la seva pràctica i ser capaç de respondre les preguntes del professor i de la resta de companys de classe.

\subsection{Presencial, semipresencial i a distància.}

Una programació didàctica ha de ser prou àgil per adaptar-se als diferents escenaris de presencialitat.
Per motius inesperats, és possible que part o la totalitat dels alumnes no puguin assistir a part o la totalitat de les classes presencials.
Per aquest motiu s'ha dissenyat un curs que permet el seu seguiment en els escenaris semipresencial i d'ensenyament a distància.

\subsection{Recursos materials.}

L'ordre EDU/392/2010 per la que s'estableix el currículum del cicle formatiu de Grau Superior corresponent al títol de Tècnic Superior en Administració de Sistemes Informàtics en Xarxa, en el seu annex IV, especifica els espais i equipaments mínims per al cicle.

A la pràctica, es necessita un ordinador amb connexió a internet i prou potent per poder utilitzar un navegador modern. També un compte de \emph{google workspace for education} per a cada alumne. S'utilitzarà google classroom com a plataforma d'e-learning i google cloudshell per a l'execució de les pràctiques. És desitjable que els ordinadors puguin executar contenidors i màquines virtuals per tenir una alternativa a google cloudshell. Per a les proves en paper es necessitarà un bolígraf.

També es necessita un projector i una pissarra.

En cas de semipresencialitat o ensenyament a distància s'emprarà també una eina de comunicació addicional, com per exemple el discord.

\section{Activitats coherents amb la metodologia.}
\label{sec:activitats_coherents}

Tal com hem dit anteriorment, l'objectiu metodològic és que els alumnes siguin protagonistes de l'assignatura.
En aquest sentit, s'han preparat pràctiques que són coherents amb aquesta metodologia i són obertes perquè l'alumne pugui prendre, en part, la iniciativa.

Les pràctiques que es plantegen normalment tenen dues parts.
La primera part és formada per una sèrie de requisits plantejats pel professor i que la pràctica ha de complir. 
Aquesta part serveix per aprovar i per obtenir el 50\% de la nota.
La segona part de la pràctica és oberta i els alumnes, seguint el seu bon criteri, han d'anar més enllà d'allò que els demana el professor.
Explorar possibilitats que no s'encabeixen, o tan sols es suggereixen, a l'enunciat de la pràctica.
Aquesta segona part de la pràctica és imprevisible per al professor i els alumnes la completaran en funció dels seus coneixements adquirits fora de la classe de l'assignatura, els seus interessos, etc.
La segona part de la pràctica permet als alumnes optar a un 50\% addicional de la nota i, per tant, treure bona nota.

S'ha de reconèixer que aquest plantejament sovint topa, al principi, amb les expectatives de l'alumnat.
A l'inici del curs és natural que els alumnes arribin amb una mentalitat de fer el mínim possible i treure la màxima nota.
``Com pot ser que no tengui la nota màxima si he fet els requisits mínims que em demana l'enunciat?'' és una pregunta habitual a principi de curs.
A la llarga, aquells alumnes amb més iniciativa agraeixen la llibertat de tenir una pràctica oberta i assumeixen el repte de fer propostes que els permetin millorar la seva pràctica per obtenir bones notes.
És natural consultar abans amb el professor. 
Els alumnes preguntaran al professor, per exemple ``Creus que és una bona idea ampliar la pràctica tractant aquest aspecte que no hem vist a classe?''.
L'objectiu final és que els alumnes siguin els protagonistes i els que tenguin la iniciativa en el procés d'ensenyament-aprenentatge.

De la mateixa manera, el professor ha d'estar preparat per fer concessions, sempre que siguin raonables, per adaptar el curs a allò que desitgen els alumnes.
També hi ha certa flexibilitat quant al ritme de l'assignatura. 
Si els alumnes estan molt engrescats en una pràctica, o tenen especial dificultat, s'ajustaran els terminis d'entrega en uns límits raonables.
La regla d'or és que si treballen durant tota la classe i no poden acabar la tasca encomanada, el professor pot concedir més temps per acabar-la.
Evidentment, els alumnes no poden demanar allargar els terminis d'entrega si no aprofiten bé el temps de classe.

S'intenta establir un clima de diàleg i consens per tal que la càrrega de feina de l'assignatura sigui proporcional al nombre d'hores assignades setmanalment.


\section{Distribució temporal.}
\label{sec:distribució}

Segons l'annex II de l'Ordre EDU/392/2010, de 20 de gener, per la que s'estableix el currículum del cicle formatiu de Grau Superior corresponent al títol de Tècnic Superior en Administració de Sistemes Informàtics en Xarxa, al mòdul de gestió de Bases de Dades li corresponen 170 hores de duració. A aquestes hores se li han de sumar les 90 reservades als mòduls impartits en anglès. El total és de 260 hores.

Segons el mateix annex, al mòdul li corresponen 5 hores anuals més unes altres 3 per ser en anglès. En total la suma és 8 hores setmanals.

En qualsevol cas, la distribució temporal que s'ofereix a continuació és orientativa i s'ajustarà al calendari escolar. També s'anirà adaptant durant el desenvolupament del curs per ajustar-se al grup d'alumnes i les altres circumstàncies de context.

La Taula \ref{tab:distribuciotemporal} mostra la proposta de distribució orientativa de hores entre mòduls.
Les hores indicades a la taula són hores de 55 minuts.
Això ens deixa una reserva d'hores per poder fer recuperacions a final de curs i per fer front a qualsevol inciència o retard imprevist.

\begin{table}
	\centering
\begin{tabular}{lr}
 Títol & Hores aproximades\\
 \hline
 Unitat 1: Introducció a les bases de dades. & 16\\
 Unitat 2: Introducció al model relacional. & 20  \\
 Unitat 3: Introducció a SQL 1. & 20\\
 Unitat 4: Introducció a SQL 2. & 20 \\
 Unitat 5: SQL intermig 1. & 20 \\
 Unitat 6: SQL intermig 2. & 20 \\
 Unitat 7: SQL avançat 1. & 24 \\
 Unitat 8: SQL avançat 2. & 24\\
 Unitat 9: Disseny de bases de dades. El model E-R. & 24\\
 Unitat 10: Disseny de bases de dades. E-R a model relacional. & 24 \\
 Unitat 11: Disseny de bases de dades. Normalització.& 24 \\
 Unitat 12: Temes avançats. & 24 \\
\end{tabular}
	\caption{Distribució temporal de la programació} \label{tab:distribuciotemporal}
\end{table}

Les distribució setmanal és de quatre sessions de 110 minuts a la setmana els dilluns, dimarts, dijous i divendres.

El calendari orientatiu, allò que tenim planificat fer cada dia de classe, es presenta a l'appèndix \ref{app:schedule}.

\section{Continguts, activitats i recursos per a les unitats didàctiques.}
\label{sec:unitats}

A l'hora de dissenyar un curs introductori de base de dades, hi ha dues aproximacions possibles.
La primera, potser més habitual a la formació professional, és programar el disseny de base dades (entitat relació, relacional) abans que el llenguatge SQL.
La segona alternativa, igualment vàlida, és ensenyar SQL abans que la metodologia de disseny de les bases de dades.
En aquesta programació didàctica s'ha elegit la segona opció.

Un dels motius, és que aquest és l'ordre seguit en el llibre de referència elegit per a aquesta signatura, el de Silbershatz \cite{silbershatz2020}.
Un altre motiu de pes és que els alumnes de dual necessiten saber SQL en el moment que s'incorporen al seu lloc de feina.
Un informàtic també ha de saber dissenyar bases de dades, però quan un s'incorpora al món laboral és més probable que ho faci treballant amb queries que no dissenyant bases de dades, que és una tasca que habitualment fan informàtics amb més experiència.
A més, la part de disseny de base de dades i normalització, que pot ser més abstracta, s'entén millor una vegada els alumnes ja tenen uns mesos d'experiència en bases de dades.
Durant la seva formació de SQL, els alumnes es familiaritzen amb bases de dades d'exemple, com una base de dades d'una escola, d'una empresa, etc.
Aquest coneixement previ els permet comprendre millor les unitats didàctiques relacionades amb el disseny de base de dades.

A continuació es descriuen cada una de les dotze unitats didàctiques d'aquesta programació.

  \subsection{Unitat 1: Introducció a les bases de dades.}

  \begin{itemize}
	\item Durada: 16 hores aprox. 24 Set. - 7 Oct.
	\item Objectius generals: 4, 5
	\item Competències professionals, personals i socials: 3, 4
	\item Resultats d'aprenentatge: 
		\begin{itemize}
			\item 1. Reconeix els elements de les bases de dades analitzant les seves funcions i valorant la utilitat de sistemes gestors.
				\begin{itemize}
					\item 1a) S'han analitzat els diferents sistemes lògics d'emmagatzematge i les seves funcions.
					\item 1b) S'han identificat els diferents tipus de bases de dades segons el model de dades utilitzat.
					\item 1c) S'han identificat els diferents tipus de bases de dades en funció de la ubicació de la informació.
					\item 1d) S'ha reconegut la utilitat d'un sistema gestor de bases de dades.
					\item 1e) S'ha descrit la funció de cada un dels elements d'un sistema gestor de bases de dades.
					\item 1f) S'han classificat els sistemes gestors de bases de dades.
				\end{itemize}
		\end{itemize}
  \end{itemize}

  
  \subsubsection{Continguts.}

  \begin{itemize}
	  \item Aplicacions de bases de dades
	  \item Propòsit dels sistemes de bases de dades
	  \item Vista de les dades
	  \item Llenguatges de bases de dades
	  \item Bases de dades relacionals
	  \item Disseny de bases de dades
	  \item Emmagatzemanent i consulta de dades
	  \item Gestió de transaccions
	  \item Arquitectura de bases de dades
	  \item Mineria de dades
	  \item Bases de dades especialitzades
	  \item Usuaris i administradors de bases de dades
	  \item Història dels sistemes de bases de dades
  \end{itemize}
  
  \subsubsection{Activitats.}

  \begin{itemize}
          \item Es reserven els primers i els darrers 5 minuts de cada sessió de 110 minuts per agafar els portàtils del carretó, endollar-los i engegar-los en cas que sigui necessari. O per recollir-los si és el cas. També resoldrem dubtes, recollirem l'opinió dels alumnes sobre el curs i mantendrem un diari de classe.
	  \item 50 minuts. Benvinguda i presentació.
	  \item 50 minuts. Valoració dels coneixements previs. Què en sabem de les bases de dades?
	  \item 100 minuts. Diapositives.
	  \item 300 minuts. Tasca. Introduction to Databases.
		  (\faGraduationCap Resultats d'aprenentatge i criteris d'avaluació) 1a, 1b, 1c, 1d, 1e, 1f.
	  \item 100 minuts. Tasca. The Role of a DBA.
		  (\faGraduationCap Resultats d'aprenentatge i criteris d'avaluació) 1d, 1e, 1f.
	  \item 100 minuts. Repàs.
	  \item 50 minuts. Test.
	  \item 50 minuts. Correcció del test.
  \end{itemize}

  \subsection{Unitat 2: Introducció al model relacional.}

  \begin{itemize}
	\item Durada: 20 hores aprox. 8 Oct. - 27 Oct.
	\item Objectius generals: 4, 5
	\item Competències professionals, personals i socials: 3, 4
	\item Resultats d'aprenentatge: 
		\begin{itemize}
			\item 2. Dissenya models lògics normalitzats interpretant diagrames entitat / relació.
				\begin{itemize}
					\item 2b) S'han utilitzat eines gràfiques per a representar el disseny lògic.
					\item 2c) S'han identificat les taules del disseny lògic.
					\item 2d) S'han identificat els camps que formen part de les taules del disseny lògic.
					\item 2f) S'han definit els camps clau.
				\end{itemize}
		\end{itemize}
  \end{itemize}

  \subsubsection{Continguts.}

  \begin{itemize}
	  \item Estructura de les bases de dades relacionals
	  \item Esquema de bases de dades
	  \item Claus
	  \item Diagrames de l'esquema
	  \item Llenguatges de consulta relacional
	  \item Operacions relacionals
  \end{itemize}

  \subsubsection{Activitats.}

  \begin{itemize}
          \item Es reserven els primers i els darrers 5 minuts de cada sessió de 110 minuts per agafar els portàtils del carretó, endollar-los i engegar-los en cas que sigui necessari. O per recollir-los si és el cas. També resoldrem dubtes, recollirem l'opinió dels alumnes sobre el curs i mantendrem un diari de classe.
          \item 100 minuts. Diapositives 2-1 Introduction to the Relational Model.
	  \item 150 minuts. Tasca 2-1 Installing a Relational Database.
	  \item 150 minuts. Tasca 2-2 Keys. (\faGraduationCap Resultats d'aprenentatge i criteris d'avaluació) 2f
	  \item 150 minuts. Tasca 2-3 Relational Diagram. (\faGraduationCap Resultats d'aprenentatge i criteris d'avaluació) 2b, 2c, 2d
	  \item 150 minuts. Tasca 2-4 Relational Algebra.
	  \item 100 minuts. Repàs.
	  \item 100 minuts. Test 2 Introduction to the Relational Model.
	  \item 100 minuts. Correcció del test.
  \end{itemize}

  \subsection{Unitat 3: Introducció a SQL 1.}

  \begin{itemize}
	\item Durada: 20 hores aprox. 28 Oct. - 15 Nov.
	\item Objectius generals: 5
	\item Competències professionals, personals i socials: 4
	\item Resultats d'aprenentatge: 
		\begin{itemize}
			\item 3. Realitza el disseny físic de bases de dades utilitzant assistents, eines gràfiques i el llenguatge de definició de dades.
				\begin{itemize}
					\item 3a) S'han definit les estructures físiques d'emmagatzematge.
					\item 3b) S'han creat taules.
					\item 3c) S'han seleccionat els tipus de dades adequats.
					\item 3d) S'han definit els camps clau en les taules.
				\end{itemize}
			\item 4. Consulta la informació emmagatzemada manejant assistents, eines gràfiques i el llenguatge de manipulació de dades.
				\begin{itemize}
					\item 4a) S'han identificat les eines i sentències per realitzar consultes.
					\item 4b) S'han realitzat consultes simples sobre una taula.
				\end{itemize}
			\item 5. Modifica la informació emmagatzemada utilitzant assistents, eines gràfiques i el llenguatge de manipulació de dades.
				\begin{itemize}
					\item 5a) S'han identificat les eines i sentències per a modificar el contingut de la base de dades.
					\item 5b) S'han inserit, esborrat i actualitzat dades en les taules.
				\end{itemize}
			\item 6. Executa tasques d'assegurament de la informació, analitzant-les i aplicant mecanismes de salvaguarda i transferència.
				\begin{itemize}
					\item 6a) S'han identificat eines gràfiques i en línia de comandes per a l'administració de còpies de seguretat.
					\item 6b) S'han realitzat còpies de seguretat.
					\item 6c) S'han restaurat còpies de seguretat.
					\item 6d) S'han identificat les eines per a importar i exportar dades.
					\item 6e) S'han exportat dades a diversos formats.
				\end{itemize}
		\end{itemize}
  \end{itemize}

  \subsubsection{Continguts.}
  \begin{itemize}
	  \item Introducció al llenguatge SQL
	  \item Definició de dades amb SQL
	  \item Estructura bàsica d'una consulta SQL
	  \item Operacions bàsiques
	  \item Operacions de conjunts
  \end{itemize}

  \subsubsection{Activitats.}

  \begin{itemize}
          \item Es reserven els primers i els darrers 5 minuts de cada sessió de 110 minuts per agafar els portàtils del carretó, endollar-los i engegar-los en cas que sigui necessari. O per recollir-los si és el cas. També resoldrem dubtes, recollirem l'opinió dels alumnes sobre el curs i mantendrem un diari de classe.
	  \item 100 minuts. Diapositives 3-1 Introduction to SQL 1.
	  \item 150 minuts. Tasca 3-1 Create a Database. (\faGraduationCap Resultats d'aprenentatge i criteris d'avaluació) 3a, 3b, 3c, 3d.
	  \item 150 minuts. Tasca 3-2 A Dockerized Database. (\faGraduationCap Resultats d'aprenentatge i criteris d'avaluació) 6a, 6b, 6c, 6d, 6e.
	  \item 150 minuts. Tasca 3-3 School Database Script. (\faGraduationCap Resultats d'aprenentatge i criteris d'avaluació) 4a, 4b, 5a, 5b, 6a, 6b, 6d, 6e.
	  \item 150 minuts. Tasca 3-4 Set Operators.
	  \item 100 minuts. Repàs.
	  \item 100 minuts. Test 3 Introduction to SQL 1.
	  \item 100 minuts. Correcció del test.
  \end{itemize}

  \subsection{Unitat 4: Introducció a SQL 2.}

  \begin{itemize}
	\item Durada: 20 hores aprox. 16 Nov. - 2 Des.
	\item Objectius generals: -
	\item Competències professionals, personals i socials: -
	\item Resultats d'aprenentatge: 
		\begin{itemize}
			\item 4. Consulta la informació emmagatzemada manejant assistents, eines gràfiques i el llenguatge de manipulació de dades.
				\begin{itemize}
					\item 4c) S'han realitzat consultes que generen valors de resum.
					\item 4f) S'han realitzat consultes amb subconsultes.
					\item 4g) S'han valorat els avantatges i inconvenients de les diferents opcions vàlides per dur a terme una consulta determinada.
				\end{itemize}
			\item 5. Modifica la informació emmagatzemada utilitzant assistents, eines gràfiques i el llenguatge de manipulació de dades.
				\begin{itemize}
					\item 5c) S'ha inclòs en una taula la informació resultant de l'execució d'una consulta.
				\end{itemize}
		\end{itemize}
  \end{itemize}

  \subsubsection{Continguts.}
  \begin{itemize}
	  \item Valors nuls
	  \item Funcions resum
	  \item Subconsultes
	  \item Modificacions de dades amb subconsultes
  \end{itemize}
  
  \subsubsection{Activitats.}

  \begin{itemize}
          \item Es reserven els primers i els darrers 5 minuts de cada sessió de 110 minuts per agafar els portàtils del carretó, endollar-los i engegar-los en cas que sigui necessari. O per recollir-los si és el cas. També resoldrem dubtes, recollirem l'opinió dels alumnes sobre el curs i mantendrem un diari de classe.
	  \item 100 minuts. Diapositives 4-1 Introduction to SQL 2.
	  \item 200 minuts. Material 4-1 Queries MariaDB. (\faGraduationCap Resultats d'aprenentatge i criteris d'avaluació) 4f.
	  \item 200 minuts. Material 4-2 Queries MariaDB. (\faGraduationCap Resultats d'aprenentatge i criteris d'avaluació) 4c, 4g, 5c.
	  \item 200 minuts. Material 4-3 Northwind.
	  \item 100 minuts. Repàs.
	  \item 100 minuts. Test 4 Introduction to SQL 2.
	  \item 100 minuts. Correcció del test.
  \end{itemize}

  \subsection{Unitat 5: SQL intermig 1.}
  
  \begin{itemize}
	\item Durada: 20 hores aprox. 3 Des. - 21 Des.
	\item Objectius generals: 4, 5
	\item Competències professionals, personals i socials: 3, 4, 13
	\item Resultats d'aprenentatge: 
		\begin{itemize}
			\item 4. Consulta la informació emmagatzemada manejant assistents, eines gràfiques i el llenguatge de manipulació de dades.
				\begin{itemize}
					\item 4d) S'han realitzat consultes sobre el contingut de diverses taules mitjançant composicions internes.
					\item 4e) S'han realitzat consultes sobre el contingut de diverses taules mitjançant composicions externes.
				\end{itemize}
			\item 5. Modifica la informació emmagatzemada utilitzant assistents, eines gràfiques i el llenguatge de manipulació de dades.
				\begin{itemize}
					\item 5f) S'ha reconegut el funcionament de les transaccions.
					\item 5g) S'han anul·lat parcialment o totalment els canvis produïts per una transacció.
					\item 5h) S'han identificat els efectes de les diferents polítiques de bloqueig de registres.
				\end{itemize}
			\item 6. Executa tasques d'assegurament de la informació, analitzant-les i aplicant mecanismes de salvaguarda i transferència.
				\begin{itemize}
					\item 6g) S'ha interpretat correctament la informació subministrada pels missatges d'error i els fitxers de registre.
				\end{itemize}
		\end{itemize}
  \end{itemize}

  \subsubsection{Continguts.}

  \begin{itemize}
	  \item Join
	  \item Vistes
	  \item Transaccions
  \end{itemize}

  \subsubsection{Activitats.}

  \begin{itemize}
          \item Es reserven els primers i els darrers 5 minuts de cada sessió de 110 minuts per agafar els portàtils del carretó, endollar-los i engegar-los en cas que sigui necessari. O per recollir-los si és el cas. També resoldrem dubtes, recollirem l'opinió dels alumnes sobre el curs i mantendrem un diari de classe.
	  \item 100 minuts. Diapositives 5-1 Intermediate SQL 1.
	  \item 100 minuts. Tasca 5-1 Dockerized phpMyAdmin.
	  \item 100 minuts. Material 5-1 Join. (\faGraduationCap Resultats d'aprenentatge i criteris d'avaluació) 4d, 4e.
	  \item 100 minuts. Material 5-2 View.
	  \item 100 minuts. Material 5-3 Transactions 1. (\faGraduationCap Resultats d'aprenentatge i criteris d'avaluació) 5f, 5g.
	  \item 100 minuts. Material 5-4 Transactions 2. (\faGraduationCap Resultats d'aprenentatge i criteris d'avaluació) 6g.
	  \item 100 minuts. Tasca 5-2 Two-phase locking and snapshop isolation. (\faGraduationCap Resultats d'aprenentatge i criteris d'avaluació) 5h.
	  \item 100 minuts. Repàs.
	  \item 100 minuts. Test 5 Intermediate SQL 1.
	  \item 100 minuts. Correcció del test.
  \end{itemize}

  \subsection{Unitat 6: SQL intermig 2.}

  \begin{itemize}
	\item Durada: 20 hores aprox. 10 Gen. - 27 Gen.
	\item Objectius generals: 5
	\item Competències professionals, personals i socials: 4
	\item Resultats d'aprenentatge: 
		\begin{itemize}
			\item 2. Dissenya models lògics normalitzats interpretant diagrames entitat / relació.
				\begin{itemize}
					\item 2g) S'han aplicat les regles d'integritat.
				\end{itemize}
			\item 5. Modifica la informació emmagatzemada utilitzant assistents, eines gràfiques i el llenguatge de manipulació de dades.
				\begin{itemize}
					\item 5d) S'han adoptat mesures per a mantenir la integritat i consistència de la informació.
				\end{itemize}
		\end{itemize}
  \end{itemize}
  
  \subsubsection{Continguts.}
  \begin{itemize}
	  \item Restriccions d'integritat
	  \item Tipus de dades
	  \item Autorització
  \end{itemize}

  \subsubsection{Activitats.}
  \begin{itemize}
          \item Es reserven els primers i els darrers 5 minuts de cada sessió de 110 minuts per agafar els portàtils del carretó, endollar-los i engegar-los en cas que sigui necessari. O per recollir-los si és el cas. També resoldrem dubtes, recollirem l'opinió dels alumnes sobre el curs i mantendrem un diari de classe.
	  \item 100 minuts. Diapositives 6-1 Intermediate SQL 1.
	  \item 100 minuts. Material 6-1 Referential integrity and data types.
	  \item 100 minuts. Tasca 6-1 Integrity Constraints.(\faGraduationCap Resultats d'aprenentatge i criteris d'avaluació) 2g, 5d. 
	  \item 100 minuts. Material 6-2 Index Importance.
	  \item 100 minuts. Material 6-3 Authorization.
	  \item 100 minuts. Material 6-4 Roles.
	  \item 100 minuts. Material 6-5 Dates.
	  \item 100 minuts. Repàs.
	  \item Material 6-6 Database for Test 6.
	  \item 100 minuts. Test 6 Intermediate SQL 2.
	  \item 100 minuts. Correcció del test.
  \end{itemize}
  

  \subsection{Unitat 7: SQL avançat 1.}

  \begin{itemize}
	\item Durada: 24 hores aprox. 28 Gen. - 17 Feb.
	\item Objectius generals: 4, 5
	\item Competències professionals, personals i socials: 3, 4, 13
	\item Resultats d'aprenentatge: -
  \end{itemize}
  
  \subsubsection{Continguts.}
  \begin{itemize}
	  \item Sentències SQL en una aplicació JAVA
  \end{itemize}

  \subsubsection{Activitats.}
  \begin{itemize}
          \item Es reserven els primers i els darrers 5 minuts de cada sessió de 110 minuts per agafar els portàtils del carretó, endollar-los i engegar-los en cas que sigui necessari. O per recollir-los si és el cas. També resoldrem dubtes, recollirem l'opinió dels alumnes sobre el curs i mantendrem un diari de classe.
	  \item 100 minuts. Material 7-1 SQL statements in a java application.
	  \item 900 minuts. Tasca 7-1 JDBC Project.
	  \item 200 minuts. Tasca 7-2 JDBC Project Video Explanation and Demonstration.
  \end{itemize}

  \subsection{Unitat 8: SQL avançat 2.}

  \begin{itemize}
	\item Durada: 24 hores aprox. 18 Feb. - 14 Març
	\item Objectius generals: 4, 5
	\item Competències professionals, personals i socials: 3, 4, 13
	\item Resultats d'aprenentatge: 
		\begin{itemize}
			\item 5. Modifica la informació emmagatzemada utilitzant assistents, eines gràfiques i el llenguatge de manipulació de dades.
				\begin{itemize}
					\item 5e) S'han dissenyat guions de sentències per dur a terme tasques complexes.
				\end{itemize}
		\end{itemize}
  \end{itemize}

  \subsubsection{Continguts.}
  \begin{itemize}
	  \item Cursors
	  \item Procediments
	  \item Disparadors
  \end{itemize}

  \subsubsection{Activitats.}
  \begin{itemize}
          \item Es reserven els primers i els darrers 5 minuts de cada sessió de 110 minuts per agafar els portàtils del carretó, endollar-los i engegar-los en cas que sigui necessari. O per recollir-los si és el cas. També resoldrem dubtes, recollirem l'opinió dels alumnes sobre el curs i mantendrem un diari de classe.
	  \item 100 minuts. Material 8-1 PostgreSQL.
	  \item 300 minuts. Material 8-2 PostgreSQL PL/pgSQL. (\faGraduationCap Resultats d'aprenentatge i criteris d'avaluació) 5e.
	  \item 200 minuts. Tasca 8-1 Create a Function.
	  \item 200 minuts. Tasca 8-2 Function Defense.
	  \item 200 minuts. Tasca 8-3 Create a Cursor, a Stored Procedure and a Trigger.
	  \item 200 minuts. Tasca 8-4 Defend your Cursor, Procedure and Trigger.
  \end{itemize}

  \subsection{Unitat 9: Disseny de bases de dades. El model E-R.}

  \begin{itemize}
	\item Durada: 24 hores aprox. 17 Març - 4 Abr.
	\item Objectius generals: 5
	\item Competències professionals, personals i socials: 4
	\item Resultats d'aprenentatge: 
		\begin{itemize}
			\item 2. Dissenya models lògics normalitzats interpretant diagrames entitat / relació.
				\begin{itemize}
					\item 2a) S'ha identificat el significat de la simbologia pròpia dels diagrames entitat / relació.
					\item 2b) S'han utilitzat eines gràfiques per a representar el disseny lògic.
				\end{itemize}
		\end{itemize}
  \end{itemize}

  \subsubsection{Continguts.}
  \begin{itemize}
	  \item Model entitat-relació
	  \item Diagrama entitat-relació
	  \item Cardinalitats
	  \item Entitats dèbils
	  \item E-R extès, generalització i especialització
  \end{itemize}

  \subsubsection{Activitats.}
  \begin{itemize}
          \item Es reserven els primers i els darrers 5 minuts de cada sessió de 110 minuts per agafar els portàtils del carretó, endollar-los i engegar-los en cas que sigui necessari. O per recollir-los si és el cas. També resoldrem dubtes, recollirem l'opinió dels alumnes sobre el curs i mantendrem un diari de classe.
	  \item 100 minuts. Diapositives 9-1.
	  \item 150 minuts. Material 9-1 E-R Diagram Tool.
	  \item 250 minuts. Material 9-2 E-R Exercises.
	  \item 500 minuts. Tasca 9-1 Create an ERD. (\faGraduationCap Resultats d'aprenentatge i criteris d'avaluació) 2a, 2b.
	  \item 200 minuts. Tasca 9-2 Defend your ERD.
  \end{itemize}

  Nota: És probable que el temps dedicat a cada activitat s'hagi de dividir per dos a causa del fet que els estudiants estaran treballant ja a l'empresa.

  \subsection{Unitat 10: Disseny de bases de dades. E-R a model relacional.}

  \begin{itemize}
	\item Durada: 24 hores aprox. 5 Abr. - 5 Maig
	\item Objectius generals: 5
	\item Competències professionals, personals i socials: 4
	\item Resultats d'aprenentatge: 
		\begin{itemize}
			\item 2. Dissenya models lògics normalitzats interpretant diagrames entitat / relació.
				\begin{itemize}
					\item 2c) S'han identificat les taules del disseny lògic.
					\item 2d) S'han identificat els camps que formen part de les taules del disseny lògic.
					\item 2e) S'han identificat les relacions entre les taules del disseny lògic.
					\item 2f) S'han definit els camps clau.
					\item 2g) S'han aplicat les regles d'integritat.
					\item 2i) S'han identificat i documentat les restriccions que no poden plasmar-se en el disseny lògic.
				\end{itemize}
			\item 3. Realitza el disseny físic de bases de dades utilitzant assistents, eines gràfiques i el llenguatge de definició de dades.
				\begin{itemize}
					\item 3e) S'han implantat totes les restriccions reflectides en el disseny lògic.
					\item 3f) S'ha verificat mitjançant un conjunt de dades de prova que la implementació s'ajusta al model.
					\item 3g) S'han utilitzat assistents i eines gràfiques.
					\item 3h) S'ha utilitzat el llenguatge de definició de dades.
					\item 3i) S'ha definit i documentat el diccionari de dades.
				\end{itemize}
		\end{itemize}
  \end{itemize}

  \subsubsection{Continguts.}
  \begin{itemize}
	\item Reducció a esquemes relacionals
  \end{itemize}

  \subsubsection{Activitats.}
  \begin{itemize}
          \item Es reserven els primers i els darrers 5 minuts de cada sessió de 110 minuts per agafar els portàtils del carretó, endollar-los i engegar-los en cas que sigui necessari. O per recollir-los si és el cas. També resoldrem dubtes, recollirem l'opinió dels alumnes sobre el curs i mantendrem un diari de classe.
	  \item 100 minuts. Diapositives 10-1.
	  \item 300 minuts. Material 10-1 E-R to relational exercises.
	  \item 600 minuts. Tasca 10-1 ERM to Relational Model. (\faGraduationCap Resultats d'aprenentatge i criteris d'avaluació) 2c, 2d, 2e, 2f, 2g, 2i, 3e, 3f, 3g, 3h, 3i.
	  \item 200 mintutes. Tasca 10-2 Defend your Relational Model.
  \end{itemize}

  Nota: És probable que el temps dedicat a cada activitat s'hagi de dividir per dos a causa del fet que els estudiants estaran treballant ja a l'empresa.

  \subsection{Unitat 11: Disseny de bases de dades. Normalització.}

  \begin{itemize}
	\item Durada: 24 hores aprox. 6 Maig - 26 Maig
	\item Objectius generals: 5
	\item Competències professionals, personals i socials: 4
	\item Resultats d'aprenentatge: 
		\begin{itemize}
			\item 2. Dissenya models lògics normalitzats interpretant diagrames entitat / relació.
				\begin{itemize}
					\item h) S'han aplicat les regles de normalització fins a un nivell adequat.
				\end{itemize}
		\end{itemize}
  \end{itemize}

  \subsubsection{Continguts.}
  \begin{itemize}
	\item Dependència funcional
	\item Dependència funcional completa
	\item Primera forma normal
	\item Segona forma normal
	\item Tercera forma normal
	\item Forma normal Boyce-Codd
  \end{itemize}

  \subsubsection{Activitats.}
  \begin{itemize}
          \item Es reserven els primers i els darrers 5 minuts de cada sessió de 110 minuts per agafar els portàtils del carretó, endollar-los i engegar-los en cas que sigui necessari. O per recollir-los si és el cas. També resoldrem dubtes, recollirem l'opinió dels alumnes sobre el curs i mantendrem un diari de classe.
	  \item 100 minuts. Diapositives 11-1(Simplificades).
	  \item Diapositives 11-2.
	  \item 300 minuts. Material 11-1 Exercises.
	  \item 100 minuts. Material 11-2 Example of Solution.
	  \item 300 minuts. Tasca 11-1 Normalization. (\faGraduationCap Resultats d'aprenentatge i criteris d'avaluació) 2h.
	  \item 200 minuts. Tasca 11-2 Defend your Normalization.
	  \item 200 minuts. Tasca 11-3 Normalization Script.
  \end{itemize}


  Nota: És probable que el temps dedicat a cada activitat s'hagi de dividir per dos a causa del fet que els estudiants estaran treballant ja a l'empresa.

  \subsection{Unitat 12: Temes avançats.}

  \begin{itemize}
	\item Durada: 24 hores aprox. 27 Maig - 16 Juny
	\item Objectius generals: 4, 5, 13
	\item Competències professionals, personals i socials: 3, 4, 13
	\item Resultats d'aprenentatge: 
		\begin{itemize}
			\item 6. Executa tasques d'assegurament de la informació, analitzant-les i aplicant mecanismes de salvaguarda i transferència.
				\begin{itemize}
					\item 6a) S'han identificat eines gràfiques i en línia de comandes per a l'administració de còpies de seguretat.
					\item 6b) S'han realitzat còpies de seguretat.
					\item 6c) S'han restaurat còpies de seguretat.
					\item 6d) S'han identificat les eines per a importar i exportar dades.
					\item 6e) S'han exportat dades a diversos formats.
					\item 6f) S'han importat dades amb diferents formats.
					\item 6g) S'ha interpretat correctament la informació subministrada pels missatges d'error i els fitxers de registre.
					\item 6h) S'ha transferit informació entre sistemes gestors.
				\end{itemize}
		\end{itemize}
  \end{itemize}

  \subsubsection{Continguts.}
  \begin{itemize}
	\item Còpies de seguretat i restauració
	\item Importació i exportació de dades
	\item Missatges d'error i fitxers de registre
	\item Transferència de dades entre sistemes gestors
	\item Bases de dades NoSQL
  \end{itemize}

  \subsubsection{Activitats.}
  \begin{itemize}
          \item Es reserven els primers i els darrers 5 minuts de cada sessió de 110 minuts per agafar els portàtils del carretó, endollar-los i engegar-los en cas que sigui necessari. O per recollir-los si és el cas. També resoldrem dubtes, recollirem l'opinió dels alumnes sobre el curs i mantendrem un diari de classe.
	  \item 600 minuts. Tasca 12-1 Advanced Topics. (\faGraduationCap Resultats d'aprenentatge i criteris d'avaluació) 6a, 6b, 6c, 6d, 6e, 6f, 6g, 6h.
	  \item 600 minuts. Material 12-1 NoSQL.
  \end{itemize}

  Nota: És probable que el temps dedicat a cada activitat s'hagi de dividir per dos a causa del fet que els estudiants estaran treballant ja a l'empresa.


\section{Mesures per a l'atenció a la diversitat.}
\label{sec:diversitat}

Per ser un ensenyament post-obligatori no es contempla la possibilitat de prendre mesures d'adaptació curricular significatives. Es prendran mesures que no impliquin una modificació substancial del currículum amb l'assessorament del departament d'orientació.

Algunes mesures habituals poden ser cedir a les persones amb necessitats especials aquells llocs de la classe que els siguin més favorables o adaptar els temps de les proves a les seves necessitats.

A més, es presentaran tasques obertes on cada alumne podrà completar-les segons les seves possibilitats i situació personal.

Excepte en els exàmens, s'afavorirà el treball en equip de tal manera que les fortaleses d'uns puguin compensar les mancances d'uns altres. S'afavorirà un clima de col·laboració on les diferències siguin una oportunitat i no un obstacle.

Les persones amb necessitats especials sovint necessiten una atenció personalitzada, però a la pràctica no disposem de personal de suport a l'aula per atendre-les.
La solució és que siguin els mateixos companys de classe els que ajudin a qui ho necessita.
Quan assignam una tasca, qualcuns alumnes són capaços d'acabar molt abans que els altres per diferents motius.
Demanarem a aquests alumnes que donin un cop de mà a aquells que vagin més lents.

Aquesta estratègia té múltiples avantatges. 
Per una part, mitiga el problema que suposa que uns acabin molt abans que els altres.
A més, els alumnes amb necessitats especials poden rebre un tracte personalitzat i dedicat per part d'un company.
Aquest company pot explicar les coses de manera més propera i diferent de com ho ha fet el professor per ajudar a entendre conceptes.
Els avantatges no són només per a la persona ajudada, sinó també per a la que ajuda.
L'alumne que ajuda als companys consolida els seus propis coneixements i amb això obtindrà també millors notes.

A més, a l'aula trobam persones amb diferents fortaleses i debilitats.
Els que un dia són ajudats en una altra ocasió potser seran ajudants i així es canviaran els rols.
Finalment, es comença a treballar amb un model que és molt habitual en el món de les tecnologies de la informació anomenat ``meritocràcia''.
En una meritocràcia, aquells que més aporten al projecte es guanyen el respecte dels companys i la seva veu és escoltada amb especial atenció a l'hora de prendre decisions.

\section{Recursos i materials.}
\label{sec:recursos_i_materials}

El material inclou un seguit d'activitats de creació pròpia, com per exemple tasques i examens.

A més, és de gran utilitat tenir un llibre de referència per al curs.
Per una banda, ajuda al professor a preparar millor el curs.
Per una altra, és un recurs addicional per a l'alumne per a poder seguir el curs de manera més autònoma.
Un llibre d'interès és ``Gestión de Bases de Datos'' de López i de Castro \cite{montalban2014}.
L'avantatge d'aquest llibre és que s'ajusta molt al currículum oficial.

El llibre que s'ha adoptat finalment com a referència per a aquest curs és ``Database System Concepts'' de Silbershatz, Korth i Sudarshan \cite{silbershatz2020}.
És un dels llibres de bases dades més reconeguts i s'utilitza a multitud de centres d'arreu del món.
Els diferents conceptes es van introduint amb gran rigor i claredat.
El desavantatge que té aquest llibre és que està orientat a un curs universitari i, per tant, no s'ajusta tan bé al currículum de la formació professional.
En conseqüència, en el nostre curs no veurem tot allò que és al llibre.

A més, els autors del llibre ofereixen també les diapositives que es corresponen als diferents capítols.
Aquestes diapositives han estat preparades per professors universitaris amb una llarga trajectòria en la docència de les bases de dades i són, possiblement, de les millors que existeixen.
A més, s'han anat actualitzant i corregint amb els anys, millorant encara més.
Es tracta d'un material de referència en l'ensenyament-aprenentatge de les bases de dades.
Aquestes diapositives estan disponibles a la mateixa web del llibre.

En qualcun cas, i de manera puntual, s'han preparat diapositives addicionals per complementar les mencionades anteriorment.
Ha estat en el cas de la unitat didàctica de normalització, on el rigor i la sofisticació del material universitari s'allunyen de l'aproximació més pràctica i planera de la formació professional.
En aquest cas, hem preparat un petit conjunt de diapositives per facilitar la comprensió dels conceptes als alumnes.

També s'usarà sempre que sigui possible material i tutorials disponibles a internet. 
El professor s'encarregarà de seleccionar material que sigui de qualitat i ajustat al currículum.
Moltes vegades, aquest material és una manera d'apropar-se a allò que la indústria informàtica espera dels seus treballadors.

Un exemple d'aquest material són els tutorials oficials de postgresql, que guien a l'alumne en l'aprenentatge d'aquest sistema gestor de bases de dades.

\section{Conclusió.}
\label{sec:conclusió}

Aquesta programació és una guia per a la impartició del mòdul ``Gestió de Bases de Dades'' en el CFGS d'ASIX.
Els aspectes que es volen ressaltar són un procés d'ensenyament-aprenentatge centrat en l'alumne.
L'alumne aprèn realitzant tasques en equip i llavors demostra la seva aptitud en proves individuals.
Els continguts venen marcats en part per la normativa oficial i en part per la realitat actual de la indústria informàtica.
Es pretèn que els alumnes aprenguin els conceptes fonamentals i adquireixin les competències necessàries per fer front a un entorn tecnològic canviant.
Volem formar treballadors de les tecnologies de la informació aptes i, al mateix temps, persones que puguin viure en plenitud en la nostra societat:
la societat de la informació i les comunicacions.



\begin{thebibliography}{9}

\bibitem{coll2017} 
J. Coll Amengual
\textit{La programació didàctica a ESO i Batxillerat}. 

\bibitem{montalban2014} 
I. López Montalbán, M. de Castro Vázquez
\textit{Gestión de Bases de Datos}. 
Editorial Garceta

\bibitem{silbershatz2020} 
A. Silbershatz, H. F. Korth i S. Sudarshan 
\textit{Database System Concepts}. 
McGraw Hill Education

\end{thebibliography}


\appendix

\section{Calendari orientatiu.}
\label{app:schedule}

\paragraph*{Tentative Schedule:}
\begin{center}
\begin{calendar}{9/20/2021}{39} % Semester starts on 9/24/2021 and last for 39
                    % weeks
\setlength{\calboxdepth}{.3in}
\MTRFClass
% schedule
\caltexton{1}{1.1 Welcome. Presentation.}
\caltextnext{1.2 Slides}
\caltextnext{1.3 Assignment 1}
\caltextnext{1.4 Slides}
\caltextnext{1.5 Assignment 2}
\caltextnext{1.6 Review}
\caltextnext{1.7 Test}
\caltextnext{1.8 Test correction}
\caltextnext{2.1 Slides}
\caltextnext{2.2 Assignment 2-1}
\caltextnext{2.3 Slides}
\caltextnext{2.4 Assignment 2-2}
\caltextnext{2.5 Assignment 2-3}
\caltextnext{2.6 Assignment 2-4}
\caltextnext{2.7 Assignment 2-4}
\caltextnext{2.8 Review}
\caltextnext{2.9 Test}
\caltextnext{2.10 Test correction}
\caltextnext{3.1 Slides}
\caltextnext{3.2 Assignment 3-1}
\caltextnext{3.3 Assignment 3-1}
\caltextnext{3.4 Assignment 3-2}
\caltextnext{3.5 Assignment 3-2}
\caltextnext{3.6 Assignment 3-3}
\caltextnext{3.7 Assignment 3-3}
\caltextnext{3.8 Review}
\caltextnext{3.9 Test}
\caltextnext{3.10 Test correction}
\caltextnext{4.1 Slides}
\caltextnext{4.2 Material 4-1}
\caltextnext{4.3 Material 4-1}
\caltextnext{4.4 Material 4-2}
\caltextnext{4.5 Material 4-2}
\caltextnext{4.6 Material 4-3}
\caltextnext{4.7 Material 4-3}
\caltextnext{4.8 Review}
\caltextnext{4.9 Test}
\caltextnext{4.10 Test correction}
\caltextnext{5.1 Slides}
\caltextnext{5.2 Assignment 5-1}
\caltextnext{5.3 Material 5-1}
\caltextnext{5.4 Material 5-2}
\caltextnext{5.5 Material 5-3}
\caltextnext{5.6 Material 5-4}
\caltextnext{5.7 Assignment 5-2}
\caltextnext{5.8 Review}
\caltextnext{5.9 Test}
\caltextnext{5.10 Test correction}
\caltextnext{6.1 Slides}
\caltextnext{6.2 Material 6-1}
\caltextnext{6.3 Assignment 6-1}
\caltextnext{6.4 Material 6-2}
\caltextnext{6.5 Matreial 6-3}
\caltextnext{6.6 Material 6-4}
\caltextnext{6.7 Material 6-5}
\caltextnext{6.8 Review}
\caltextnext{6.9 Test}
\caltextnext{6.10 Test correction}
\caltextnext{7.1 Material 7-1}
\caltextnext{7.2 Assignment 7-1}
\caltextnext{7.3 Assignment 7-1}
\caltextnext{7.4 Assignment 7-1}
\caltextnext{7.5 Assignment 7-1}
\caltextnext{7.6 Assignment 7-1}
\caltextnext{7.7 Assignment 7-1}
\caltextnext{7.8 Assignment 7-1}
\caltextnext{7.9 Assignment 7-1}
\caltextnext{7.10 Assignment 7-2}
\caltextnext{7.11 Presentations}
\caltextnext{7.12 Presentations}
\caltextnext{8.1 Material 8-1}
\caltextnext{8.2 Material 8-2}
\caltextnext{8.3 Material 8-3}
\caltextnext{8.4 Material 8-4}
\caltextnext{8.5 Assignment 8-1}
\caltextnext{8.6 Assignment 8-1}
\caltextnext{8.7 Assignment 8-2}
\caltextnext{8.8 Assignment 8-2}
\caltextnext{8.9 Assignment 8-3}
\caltextnext{8.10 Assignment 8-3}
\caltextnext{8.11 Presentations}
\caltextnext{8.12 Presentations}
\caltextnext{9.1 Slides}
\caltextnext{9.2 Slides}
\caltextnext{9.3 Material 9-1}
\caltextnext{9.4 Assignment 9-1}
\caltextnext{9.5 Assignment 9-1}
\caltextnext{9.6 Assignment 9-1}
\caltextnext{9.7 Assignment 9-1}
\caltextnext{9.8 Assignment 9-1}
\caltextnext{9.9 Assignment 9-1}
\caltextnext{9.10 Assignment 9-1}
\caltextnext{9.11 Presentations}
\caltextnext{9.12 Presentations}
\caltextnext{10.1 Slides}
\caltextnext{10.2 Slides}
\caltextnext{10.3 Assignment 10-1}
\caltextnext{10.4 Assignment 10-1}
\caltextnext{10.5 Assignment 10-1}
\caltextnext{10.6 Assignment 10-1}
\caltextnext{10.7 Assignment 10-1}
\caltextnext{10.8 Assignment 10-1}
\caltextnext{10.9 Assignment 10-1}
\caltextnext{10.10 Assignment 10-1}
\caltextnext{10.11 Presentations}
\caltextnext{10.12 Presentations}
\caltextnext{11.1 Slides}
\caltextnext{11.2 Slides}
\caltextnext{11.3 Material 11-1}
\caltextnext{11.4 Material 11-1}
\caltextnext{11.5 Material 11-2}
\caltextnext{11.6 Assignment 11-1}
\caltextnext{11.7 Assignment 11-1}
\caltextnext{11.8 Assignment 11-1}
\caltextnext{11.9 Assignment 11-1}
\caltextnext{11.10 Presentations}
\caltextnext{11.11 Presentations}
\caltextnext{11.12 Assignment 11-3}
\caltextnext{12.1 Assignment 12-1}
\caltextnext{12.2 Assignment 12-1}
\caltextnext{12.3 Assignment 12-1}
\caltextnext{12.4 Assignment 12-1}
\caltextnext{12.5 Material 12-1}
\caltextnext{12.6 Material 12-1}
\caltextnext{12.7 Material 12-1}
\caltextnext{12.8 Material 12-1}
\caltextnext{12.9 Material 12-1}
\caltextnext{12.10 Material 12-1}
\caltextnext{12.11 Material 12-1}
\caltextnext{12.12 Material 12-1}
% ... and so on

% Holidays
\Holiday{10/12/2021}{Pilar}
\Holiday{11/1/2021}{Tots Sants}
\Holiday{12/6/2021}{Constitució}
\Holiday{12/8/2021}{Immaculada}
\Holiday{1/20/2022}{Sant Sebastià}
\Holiday{2/25/2022}{Substitucio festa local}
\Holiday{2/28/2022}{Festa escolar unificada}
\Holiday{4/14/2022}{Dijous Sant}
\Holiday{4/15/2022}{Divendres Sant}
\Holiday{5/1/2022}{Primer de maig}
\Holiday{24/6/2022}{Sant Joan}
% ... and so on

\options{9/20/2021}{\noclassday} 
\options{9/21/2021}{\noclassday} 
\options{9/22/2021}{\noclassday} 
\options{9/23/2021}{\noclassday} 
\options{12/23/2021}{\noclassday} 
\options{12/24/2021}{\noclassday}
\options{12/25/2021}{\noclassday} 
\options{12/26/2021}{\noclassday}
\options{12/27/2021}{\noclassday}
\options{12/28/2021}{\noclassday}
\options{12/29/2021}{\noclassday}
\options{12/30/2021}{\noclassday}
\options{12/31/2021}{\noclassday}
\options{1/1/2022}{\noclassday} 
\options{1/2/2022}{\noclassday}
\options{1/3/2022}{\noclassday}
\options{1/4/2022}{\noclassday}
\options{1/5/2022}{\noclassday}
\options{1/6/2022}{\noclassday}
\options{1/7/2022}{\noclassday}
\options{1/8/2022}{\noclassday}
\options{1/9/2022}{\noclassday}
\options{4/14/2022}{\noclassday}
\options{4/15/2022}{\noclassday}
\options{4/16/2022}{\noclassday}
\options{4/17/2022}{\noclassday}
\options{4/18/2022}{\noclassday}
\options{4/19/2022}{\noclassday}
\options{4/20/2022}{\noclassday}
\options{4/21/2022}{\noclassday}
\options{4/22/2022}{\noclassday}
\caltext{4/27/2011}{\textbf{Final Exam}}
\end{calendar}
\end{center}

\end{document}
